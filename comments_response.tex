% Options for packages loaded elsewhere
\PassOptionsToPackage{unicode}{hyperref}
\PassOptionsToPackage{hyphens}{url}
%
\documentclass[
]{article}
\usepackage{lmodern}
\usepackage{amssymb,amsmath}
\usepackage{ifxetex,ifluatex}
\ifnum 0\ifxetex 1\fi\ifluatex 1\fi=0 % if pdftex
  \usepackage[T1]{fontenc}
  \usepackage[utf8]{inputenc}
  \usepackage{textcomp} % provide euro and other symbols
\else % if luatex or xetex
  \usepackage{unicode-math}
  \defaultfontfeatures{Scale=MatchLowercase}
  \defaultfontfeatures[\rmfamily]{Ligatures=TeX,Scale=1}
\fi
% Use upquote if available, for straight quotes in verbatim environments
\IfFileExists{upquote.sty}{\usepackage{upquote}}{}
\IfFileExists{microtype.sty}{% use microtype if available
  \usepackage[]{microtype}
  \UseMicrotypeSet[protrusion]{basicmath} % disable protrusion for tt fonts
}{}
\makeatletter
\@ifundefined{KOMAClassName}{% if non-KOMA class
  \IfFileExists{parskip.sty}{%
    \usepackage{parskip}
  }{% else
    \setlength{\parindent}{0pt}
    \setlength{\parskip}{6pt plus 2pt minus 1pt}}
}{% if KOMA class
  \KOMAoptions{parskip=half}}
\makeatother
\usepackage{xcolor}
\IfFileExists{xurl.sty}{\usepackage{xurl}}{} % add URL line breaks if available
\IfFileExists{bookmark.sty}{\usepackage{bookmark}}{\usepackage{hyperref}}
\hypersetup{
  pdftitle={Comments response (SOFTX\_2020\_29)},
  hidelinks,
  pdfcreator={LaTeX via pandoc}}
\urlstyle{same} % disable monospaced font for URLs
\usepackage[margin=1in]{geometry}
\usepackage{graphicx,grffile}
\makeatletter
\def\maxwidth{\ifdim\Gin@nat@width>\linewidth\linewidth\else\Gin@nat@width\fi}
\def\maxheight{\ifdim\Gin@nat@height>\textheight\textheight\else\Gin@nat@height\fi}
\makeatother
% Scale images if necessary, so that they will not overflow the page
% margins by default, and it is still possible to overwrite the defaults
% using explicit options in \includegraphics[width, height, ...]{}
\setkeys{Gin}{width=\maxwidth,height=\maxheight,keepaspectratio}
% Set default figure placement to htbp
\makeatletter
\def\fps@figure{htbp}
\makeatother
\setlength{\emergencystretch}{3em} % prevent overfull lines
\providecommand{\tightlist}{%
  \setlength{\itemsep}{0pt}\setlength{\parskip}{0pt}}
\setcounter{secnumdepth}{-\maxdimen} % remove section numbering

\title{Comments response (SOFTX\_2020\_29)}
\author{}
\date{\vspace{-2.5em}}

\begin{document}
\maketitle

We first wish to thank the two reviewers for helpful comments. We have
answered them all as below. The revised sections were marked by red text
in the manuscript.

\textbf{Reviewer 1}

Overall well this is a well written paper written although there are a
few typos in the text and it would benefit from a modest reordering of
some of the text. Methodologically this is a sound piece of work. It
does not contribute to the theory of GSA but the ability to introduce a
phase-change is a useful advance compared to the eFAST routine coded in
the R sensitivity package. The ability to work with models coded in GNU
MCSim is particularly useful since there is no support for GSA within
this software. The plots that can be seen in Figure 2 neatly demonstrate
the advantage of taking replicates although for a highly parameterised
and computationally expensive model this will come with a hefty
overhead, particularly for models coded using deSolve. The paper over
really skims over the capability of the pksensi package although it does
provide a link to fuller guidance.

\textbf{Ans: We have edited the paper to address typos, as well as did
some re-ordering of the text, as described below. Due to word count
limits, we are unable to provide additional details in the text as to
the capability of the package, though as the reviewer notes, links to
fuller guidance are included.}

I have only a few minor comments.

\begin{enumerate}
\def\labelenumi{\arabic{enumi}.}
\item
  The workflow shown in figure 1 demonstrates what I would regard as
  best practice for development and calibration of a PBPK model.
  Uncertainty analysis in some form for the testing, debugging and
  assessment of the quantitative behaviour of a model over the bounds of
  the input parameters is an essential prerequisite to sensitivity
  analysis. However, whilst this is briefly covered in the text
  (beginning on line 137) the authors don't specify how the results from
  uncertainty analysis might be used to refine parameter ranges.
  Furthermore, there is no mention of uncertainty analysis in the 5 step
  routine in the illustrative examples section.\\
  \textbf{Ans: The basic idea of uncertainty analysis in pksensi is to
  examine if the data from the in-vivo experiment can be located in the
  range of model simulation output. If not, the modeler should examine
  all possible reasons that cause the problem (e.g., too narrow
  parameter range or incomplete model structure). However, detailing the
  parameter adjustment process after uncertainty analysis is out of the
  scope of this package, and additional explanation can exceed the word
  limit in the softwareX. We do, though, have an example of uncertainty
  analysis for the acetaminophen-PBPK model that was provided in the
  pksensi website
  (\url{https://nanhung.netlify.app/pksensi/articles/pbpk_apap.html}).
  Additionally, the step of uncertainty analysis had also been added to
  the examples section.}
\item
  I suggest the authors review the ordering of the text so that it
  reflects the workflow of figure 1. At present the text on uncertainty
  analysis follows the text on sensitivity analysis.\\
  \textbf{Ans: We revised and added the content from the sections
  ``Uncertainty analysis'' and ``Sensitivity analysis'' to reflect the
  workflow of figure 1.}
\item
  Paragraph starting on line 137. `The uncertainty analysis is a crucial
  modelling process within SA \ldots{}' Uncertainty analysis is the
  study of uncertainty in model outputs that arises as a consequence of
  uncertainty on model inputs. Sensitivity analysis goes a step further
  in attributing the uncertainty in outputs to individual parameters or
  interactions between parameters. Uncertainty and sensitivity analysis
  are distinct but complimentary processes.\\
  \textbf{Ans: We separate the uncertainty and sensitivity analysis into
  two subsections (2.4 \& 2.5) for clarity.}
\end{enumerate}

\textbf{Reviewer 2}

\begin{itemize}
\item
  Major Revision

  \begin{enumerate}
  \def\labelenumi{\arabic{enumi}.}
  \tightlist
  \item
    This article does not follow templates of this journal:
  \end{enumerate}

  \begin{itemize}
  \tightlist
  \item
    The words in abstract is about 222 words which is much larger than
    100 words.
  \item
    The words in main text is about 5578 words which is much larger than
    3000 words.
  \item
    Software metadata table is missing.\\
    \textbf{Ans: After double-checking the Guide for Authors in
    SoftwareX web page
    (\url{https://www.elsevier.com/journals/softwarex/2352-7110/guide-for-authors}),
    we think that there is no word restriction for the abstract.
    Overall, we removed some context make the article more concise. The
    result from textcount
    (\url{https://app.uio.no/ifi/texcount/online.php}) shows that the
    words in the manuscript are under 3000 words as below. The software
    metadata table had been added in the manuscript as well.}
  \end{itemize}
\end{itemize}

\begin{verbatim}
Word count
Words in text: 2842
Words in headers: 55
Words outside text (captions, etc.): 75
Number of headers: 18
Number of floats/tables/figures: 5
Number of math inlines: 14
Number of math displayed: 2
Subcounts:
  text+headers+captions (#headers/#floats/#inlines/#displayed)
  222+15+0 (2/0/0/0) _top_
  0+2+0 (1/0/0/0) Section: Required Metadata
  0+3+3 (1/1/0/0) Section: Current code version
  598+3+0 (1/0/0/0) Section: Motivation and significance
  72+2+0 (1/0/0/0) Section: Software description
  39+1+0 (1/0/0/0) Subsection: Installation
  402+3+0 (1/0/11/2) Subsection: Parameter matrix generation
  149+1+0 (1/0/0/0) Subsection: Modeling
  175+2+0 (1/0/0/0) Subsection: Uncertainty analysis
  122+2+0 (1/0/0/0) Subsection: Sensitivity analysis
  405+2+0 (1/0/3/0) Section: Illustrative Examples
  529+3+0 (1/0/0/0) Section: Impact and conclusions
  26+3+0 (1/0/0/0) Section: Conflict of Interest
  82+1+0 (1/0/0/0) Section: Acknowledgements
  0+4+0 (1/0/0/0) Section: Appendix A. Example code
  21+4+69 (1/3/0/0) Section: CRediT authorship contribution statement
  0+4+3 (1/1/0/0) Section: Current executable software version
\end{verbatim}

\begin{enumerate}
\def\labelenumi{\arabic{enumi}.}
\setcounter{enumi}{1}
\tightlist
\item
  Proprietary names such as R package names would be shown better in
  Gothic or bold font. `pksensi' is in bold but other package names such
  as `sensitivity' are not in bold, which is a little confusing in
  reading the text.\\
  \textbf{Ans: We revised the article and used bold font for all
  mentioned R packages.}
\end{enumerate}

\begin{itemize}
\item
  Minor Comments

  \begin{enumerate}
  \def\labelenumi{\arabic{enumi}.}
  \tightlist
  \item
    Line 187-188: Statements are hard to understand. It needs correction
    at `to further used,' and `thwe' in the following text: ``The
    corresponding ranges have to define in a list object, a generic
    vector to further used in the analysis. After the definition, thwe
    users can use rfast99() to create parameter matrix.''\\
    \textbf{Ans: We revised the description as ``The next step is to
    identify the model parameters that will be examed in global SA and
    create the parameter matrix. After parameter selection, we have to
    assign each parameter a probability distribution over which to
    evaluate, such as a uniform distribution with specified min and max.
    The users then can use \textit{\textbf{rfast99()}} with the above
    assignments (e.g., parameter names, sample number, parameter
    distribution, and its corresponding properties) to create the
    parameter matrix.''}
  \end{enumerate}
\end{itemize}

\end{document}
